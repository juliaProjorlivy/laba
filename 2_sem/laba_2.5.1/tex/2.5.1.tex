\normalsize{\bold{Цель работы:}1) измерение температурной зависимости  
коэффициента поверхностного натяжения дистиллированной воды с использованием 
известного коэффициента поверхностного натяжения спирта;  
2) определение полной поверхностной энергии  и теплоты, необходимой 
для изотермического образования единицы  поверхности жидкости  при различной 
температуре. } \\ [6pt]

\normalsizez{\bold{Оборудование:}прибор  Ребиндера  с термостатом и 
микроманометром; исследуемые жидкости; стаканы.}

\section*(Теоретические сведения и экспериментальная установка)
\normalsize{Для сферического пузырька с воздухом  внутри жидкости 
избыточное давление даётся формулой Лапласа:}
\begin{equation}
    \Delta P = P_{\text{внутри} - P_{\text{снаружи}} = \frac{2\sigma}{r}}
\end{equation}
\normalsize{Соответсвенно эта формула и будет испльзована для определения
коэффициента поверхностного натяжения жидкости.}

